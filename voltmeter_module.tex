\documentclass[11pt]{article}

%\usepackage{sectsty}
\usepackage{siunitx}
\usepackage{tabularx}
\usepackage{float}
\usepackage{graphicx}
\usepackage{mathrsfs}
\usepackage{subcaption}
\usepackage{hyperref}
\usepackage{url}
\usepackage{csquotes}
\usepackage{verbatim}
\usepackage{cite}
\usepackage{stfloats}
\usepackage{textcomp}
\usepackage{algorithm}
\usepackage{algorithmic}
\usepackage{amsmath, amsfonts}
\usepackage{cmsrb}
\usepackage[serbian]{babel}

% Margins
\topmargin=-0.45in
\evensidemargin=0in
\oddsidemargin=0in
\textwidth=6.5in
\textheight=9.0in
\headsep=0.25in

\title{MCP3651 based voltmeter module}

\begin{document}

\maketitle
\pagebreak
\tableofcontents
\pagebreak

\section{Resolution}
MCP3651 ADC has a resolution of 24 bits, however, the 7 ppm max 
INL limits this multimeters count to 146000, just shy of 5.5 digit.

Keeping drift in 1 PPM range, 3.3 uV drift is acceptable over the 
temperature range.

\section{Power supply}

\subsection{Expected load}
\subsection{Revision 1}



\section{Mounting holes}
4 mounting posts are spaced in a 80 mm by 40 mm rectangle. Plastic post 
hole width is about 2.5 mm, but to allow for a self tapping screw, the PCB 
hole must be M3.

\begin{figure}[H]
  \centering 
  \includegraphics[scale=0.4]{"./figs/screw_sizez.jpg"}
\end{figure}

Apropriate screw is has a B of 2 mm, L of 6 mm, A of 5 mm and C of 3 mm. 

\section{Input connectors}

\subsection{PCB mount connector}

\begin{figure}[H]
  \centering 
  \includegraphics[scale=0.1]{"./figs/pomona.jpg"}
\end{figure}



\href{https://www.mouser.co.uk/ProductDetail/Pomona-Electronics/73099-2?qs=B6kkDfuK7%2FA6DpEPKtHqWw%3D%3D}{Pomona connector} 
  cost around 8 EUR.


\subsection{Panel mount}

\section{Charging connector}
\subsection{Panel mount USB C}

\href{https://www.aliexpress.com/item/1005005795420370.html?spm=a2g0o.detail.pcDetailTopMoreOtherSeller.8.7952vy1gvy1gDZ&gps-id=pcDetailTopMoreOtherSeller&scm=1007.40000.327270.0&scm_id=1007.40000.327270.0&scm-url=1007.40000.327270.0&pvid=2700dcaa-0eb6-4b0e-88eb-943a58278a62&_t=gps-id:pcDetailTopMoreOtherSeller,scm-url:1007.40000.327270.0,pvid:2700dcaa-0eb6-4b0e-88eb-943a58278a62,tpp_buckets:668%232846%238115%232000&pdp_npi=4%40dis%21RSD%2122.34%2118.09%21%21%210.21%210.17%21%402101ef5e17281249438301346ebeab%2112000034381437374%21rec%21SRB%21%21ABXZ&utparam-url=scene%3ApcDetailTopMoreOtherSeller%7Cquery_from%3A}{Aliexpress panel mount USB C connector}
cost around 1.2 EUR per 10 qty.\\

This connector may not have the required resistors in order to negociate 
current demand

\href{https://www.kupujemprodajem.com/elektronika-i-komponente/moduli-za-samoizgradnju/usb-c-konektor-za-montazu/oglas/149804414?filterId=4441321394}{Kupujemprodajem konektor}

\section{Input divider}
\subsection{Resistance divider}
With single sided input impedance of 4 Meg and a required attenuation of 1:400, total
impedance from opamp input to gnd must be:

\begin{equation}
  R_{in} = \frac{4\ \si{\mega \ohm}}{399} = 10.025\ \si{\kilo \ohm}
  \label{eq:res_to_gnd}
\end{equation}

Total resistance to ground is a parallel connection of the dividing resistor,
10 \si{\mega \ohm} input offset trim pot and 10 \si{\mega \ohm} opamp input resistance.

Dividing resistor value should then equal:

\begin{equation}
  R_{div} = \frac{5 \si{\mega \ohm} \cdot 
  10.025 \si{\kilo \ohm}}{5 \si{\mega \ohm} - 10.025 \si{\kilo \ohm}} = 
  10.045 \si{\ohm}
  \label{eq:input_divider}
\end{equation}

Placing a 0.1\% resistor (10.010 \si{\kilo \ohm} max), trim pot should be around 
50 \si{\ohm}.

\subsection{Capacitance divider}
4 series 10 \si{\pico \farad } capacitors yield total 2.5 \si{\pico \farad } with a 
tolerance of 5\%. In order to achieve 1:400 attenuation, total capacitance on the 
buffers input pin should be 997.5 pF. If we account for the 5\% tolerance, 
total capacitance should be trimmable in the range for 947.62 pF to 1047.4 pF.

\href{https://www.analog.com/media/en/technical-documentation/data-sheets/AD8038_8039.pdf}{AD8039}
features a typical input capacitance of 2 pF and thus should not significantly 
impact the total capacitance.


\subsubsection{Trim potenciometers}
Available trimmable capacitor has a capacitance range from 8 pF to 50 pF, not 
enough to account for the 5\% capacitor tolerance.

These trim capacitors are expensive (5 EUR per 1 qty) and are only single turn,
limiting trim accuracy. 


\subsubsection{Varicap trimming}
In order to allow for auto-calibration, input capacitance needs to be control
voltage dependant. To achieve this, a varicap diode may be used.

This component can be trimmed manually, using a trim potentiometer.

Since varicap add a voltage dependant capacitance value, total capacitance before
the adding the varicap needs to be less then the total capacitance required. 

As 5\% capacitors are used, fixed capacitors used should equal:

\begin{equation}
  C_{placed} = \frac{C_{req}}{1.05} - C_{varicap\ min}
  \label{eq:}
\end{equation}

Parallel connection of 2 470 pF yield capacitance in the range of 893 pF to 987 pF.

Since the capacitance change required is 10\% of required capacitance, varicap 
capacitance change in the available control voltage range (0 V to 5V) must be around
100 pF.\\

Diode selection:
\begin{itemize}
  \item Cheaper varicap diode is \href{https://eu.mouser.com/ProductDetail/Toshiba/1SV324TPH3F?qs=EEns8I54Y6DPMP6VMy8m2w%3D%3D}{toshibas}
  at 0.26 EUR per 10 qty. 

  \item Better documented varicap diode is \href{https://eu.mouser.com/ProductDetail/Skyworks-Solutions-Inc/SMV1255-079LF?qs=WMHGlxXAKT8jslp0hTOZuw%3D%3D}{skyworks} 
  at 0.51 EUR per 10 qty. 
\end{itemize}

PCB needs to support both of these diode footprints.


%--Paper--
\section{Offset results}
\subsection{10x gain enabled}
-8.5 \si{\milli \volt}

\subsection{1x gain enabled}
9.2 \si{\milli \volt}

\subsection{solving}

\begin{equation}
  V_{off\_10} = 10 \cdot (2 V_{in\_off} + V_{curr\_off}) + V_{out\_off}
\end{equation}

\begin{equation}
  V_{off\_1} = 2 V_{in\_off} + V_{curr\_off} + V_{out\_off}
\end{equation}

\begin{equation}
  V_{off\_10} - V_{off\_1}  = 9 \cdot (2 V_{in\_off} + V_{curr\_off}) = -17.7 \si{\milli \volt}
\end{equation}

Total input offset is -2 \si{\milli \volt}.
Total output offset is 11.2 \si{\milli \volt}.


\section{Sources of offset}
\subsection{Input buffer}
\subsubsection{Input bias current offset}
AD8039 has an input bias current offset of $25 \si{\nano \ampere}$, across the 20 \si{\kilo \ohm}
input impedance, generates 0.5 mV of offset. Taking into account the second input buffer, maximum offset 
is 1 mV.

\subsubsection{Input voltage offset}
AD8039 has an max input voltage offset of $3 \si{\milli \volt}$.
Worst case total offset is $6 \si{\milli \volt}$.

\subsection{Output buffer}
\subsubsection{Input bias current offset}
AD8009 has an max input current offset of $150 \si{\micro \ampere}$. With an input impedance of 
100 \si{\ohm}, total offset is 15 mV. If both inputs have opposing offsets, the total offset is 30 mV

\subsubsection{Input voltage offset}
AD8009 has an max input voltage offset of $5 \si{\milli \volt}$.

\section{Output diff amp}
AD8039 load resistor is it's outputs series resistor. 

\section{Noise}

\subsection{Resistor noise}


\begin{equation}
  V_{rms} = \sqrt{4 k_B T R \Delta f}
\end{equation}

At the bandwidth of 100 \si{\mega \hertz} and temperature of 25 \si{\celsius}, noise is

\begin{equation}
  V_{rms} = 3.63 \si{\milli \volt_{RMS}}
\end{equation}

\subsection{Input buffer noise}

AD8039 has an input noise level of $8 \si{\nano \volt} / \sqrt{\si{\hertz}}$ meaning 
that total input noise rms at 100 \si{\mega \hertz} bandwidth is 80 \si{\micro \volt_{RMS}}.

\subsection{Output buffer noise}
AD8009 has an input noise level of $1.9 \si{\nano \volt} / \sqrt{\si{\hertz}}$ meaning 
that total input noise rms at 100 \si{\mega \hertz} bandwidth is 19 \si{\micro \volt_{RMS}}.


\end{document}
